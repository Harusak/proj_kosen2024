% 行数設定など行う
\makeatletter
\def\mojiparline#1{
    \newcounter{mpl}
    \setcounter{mpl}{#1}
    \@tempdima=\linewidth
    \advance\@tempdima by-\value{mpl}zw
    \addtocounter{mpl}{-1}
    \divide\@tempdima by \value{mpl}
    \advance\kanjiskip by\@tempdima
    \advance\parindent by\@tempdima
}
\makeatother
\def\linesparpage#1{
    \baselineskip=\textheight
    \divide\baselineskip by #1
}

% 余白の設定
% A4(297mm)
\setlength{\textheight}{227truemm}
\setlength{\headheight}{20truemm}
\setlength{\topskip}{20truemm}
\setlength{\headsep}{15truemm}
\setlength{\footskip}{0truemm}
\addtolength{\topmargin}{-1truein}

% セクションの番号付けの深さを設定
% 例えば3を指定すると、\subsubsubsectionまでは番号がつく
\setcounter{secnumdepth}{5}

% PDFのしおり(ブックマーク)の管理を行うためのパッケージ
\usepackage{bookmark}

% XeLaTeXのための追加の機能(多言語サポートなど)を提供するパッケージ
\usepackage{xltxtra}

% 日本語の文書をXeLaTeXで組版するためのパッケージ
\usepackage{zxjatype}

% 日本語フォント(IPAフォント)を使用するためのパッケージ
\usepackage[ipa]{zxjafont}
% ヒラギノの場合はこっち
% \usepackage[hiragino]{zxjafont}

% 図や表のキャプションのスタイルを設定するためのパッケージ
\usepackage{caption}

% 図のキャプションのフォント、ラベル、セパレーター、名前を設定
\captionsetup[figure]{labelfont=bf,labelsep=period,name=図}

% 行間の設定
\usepackage{setspace}  % 行間を設定するためのパッケージ
\setstretch{1.2}  % 1.5倍の行間

\renewcommand{\contentsname}{目次}

% 図の複数配置
% https://gedevan-aleksizde.github.io/rmarkdown-cookbook/latex-subfigure.html
\usepackage{subfig}

% 図の配置ルール https://gedevan-aleksizde.github.io/rmarkdown-cookbook/figure-placement.html#latex-配置ルールを調整する
\usepackage{float}
\let\origfigure\figure
\let\endorigfigure\endfigure
\renewenvironment{figure}[1][2] {
    \expandafter\origfigure\expandafter[H]
} {
    \endorigfigure
}

\usepackage{flafter}
\renewcommand{\topfraction}{.85}
\renewcommand{\bottomfraction}{.7}
\renewcommand{\textfraction}{.15}
\renewcommand{\floatpagefraction}{.66}
\setcounter{topnumber}{3}
\setcounter{bottomnumber}{3}
\setcounter{totalnumber}{4}